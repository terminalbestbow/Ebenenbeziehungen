\documentclass[11pt]{article}
    \title{\textbf{Vollständige Funktionsuntersuchung}}
    \author{Tom Chlupacek}
    
    \addtolength{\topmargin}{-3cm}
    \addtolength{\textheight}{3cm}
\begin{document}
\maketitle
\section{Definitionsbereich bestimmen}
Der Definitionsbereich ist meist aufgabenabhängig. Oft wird auch keiner angegeben, dann kannst du [$-\infty;\infty$] annehmen. Beachte während der Aufgabe, dass alle x-Werte im Bereich liegen. 

\section{Symmetrie prüfen}
\subsection{Achsensymmetrie}
Wenn die Funktion symmetrisch zur y-Achse ist, gilt: 
\begin{equation} f(-x) = f(x) \end{equation}
Alternativ kann mach sich auch merken: Wenn bei einer ganzrationalen Funktion alle Exponenten gerade sind, ist die Funktion achsensymmetrisch zur y-Achse.
\subsection{Punktsymmetrie}
  


\end{document}